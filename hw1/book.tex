\documentclass[14pt]{book}
\usepackage[utf8x]{inputenc}
\usepackage[T2A]{fontenc}
\usepackage[russian,english]{babel}
\usepackage{amsmath}
\usepackage{cmap}
\usepackage[a4paper,left=4cm, right=4cm, top=4cm, bottom=4cm]{geometry}
\title{Книга 1}
\author{Имя Автора1 \and Имя Автора2}
\date{\today}

\begin{document}

\maketitle

\begin{center}
    {\huge Книга что-то там...}
\end{center}

\renewcommand{\contentsname}{Содержание}
\tableofcontents
\thispagestyle{empty}

\chapter[Введение]{
\tiny{\textbf{Текст введения жирным} \textit{Текст введения курсивом} Текст введения обычным}
}

\chapter[Основная часть]{
\large{\textbf{Текст основной части жирным} \textit{Текст основной части курсивом} Текст основной части обычным}
}

\chapter[Подраздел]{
\large{\textbf{Текст подраздела жирным} \textit{Текст подраздела курсивом} Текст подраздела обычным}
}

\chapter[Заключение]{
\huge{\textbf{Текст заключения жирным} \textit{Текст заключения курсивом} Текст заключения обычным}
}
\end{document}