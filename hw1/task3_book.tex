\documentclass[14pt]{article}
\usepackage[utf8x]{inputenc}
\usepackage[T2A]{fontenc}
\usepackage[russian,english]{babel}
\usepackage{amsmath}
\usepackage{cmap}
\usepackage[a4paper,left=4cm, right=4cm, top=4cm, bottom=4cm]{geometry}
\title{Книга 1}
\author{Имя Автора1 \and Имя Автора2}
\date{\today}

\begin{document}

\maketitle

\begin{center}
    {\huge Книга что-то там...}
\end{center}
\thispagestyle{empty}

\newpage
\hfill \break
\renewcommand{\contentsname}{Содержание}
\tableofcontents
\thispagestyle{empty}

\newpage
\section{Введение}
\tiny{\textbf{Текст введения жирным} \textit{Текст введения курсивом} Текст введения обычным}

\section{Основная часть}
\large{\textbf{Текст основной части жирным} \textit{Текст основной части курсивом} Текст основной части обычным}

\subsection{Подраздел}
\large{\textbf{Текст подраздела жирным} \textit{Текст подраздела курсивом} Текст подраздела обычным}

\section{Заключение}
\huge{\textbf{Текст заключения жирным} \textit{Текст заключения курсивом} Текст заключения обычным}

\end{document}