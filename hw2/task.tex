\documentclass[14pt]{article}
\usepackage[utf8x]{inputenc}
\usepackage[T2A]{fontenc}
\usepackage[russian,english]{babel}
\usepackage{amsmath}
\usepackage{cmap}
\usepackage[a4paper, left=2cm, right=2cm, top=2cm, bottom=2cm]{geometry}

\begin{document}
 
\begin{center}
\hfill \break
\textbf{\normalsize{Министерство науки и высшего образования Российской Федерации\\
Федеральное государственное автономное образовательное\\
учреждение высшего образования}}
\\
\normalsize{\textbf{«КАЗАНСКИЙ (ПРИВОЛЖСКИЙ) ФЕДЕРАЛЬНЫЙ УНИВЕРСИТЕТ»}}\\
\hfill \break
\normalsize{ИНСТИТУТ ВЫЧИСЛИТЕЛЬНОЙ МАТЕМАТИКИ\\ И ИНФОРМАЦИОННЫХ ТЕХНОЛОГИЙ}\\
 \hfill \break
\normalsize{Кафедра прикладной математики}\\
\hfill\break
\hfill \break
\normalsize{Направление подготовки: 01.03.04 – Прикладная математика}\\
\normalsize{Профиль: Математическое моделированиея}\\
\hfill \break
\hfill \break
КУРСОВАЯ РАБОТА\\
\hfill \break
Название курсовой работы\\
\hfill \break
\hfill \break
\end{center}
 
\normalsize{\hspace{2mm}Допущено к защите \dots } \hfill \break
\hfill \break
 
\normalsize{ 
\begin{tabular}{cccc}
Зав.кафедрой & \underline{\hspace{3cm}} & должность & ФИО \\\\
Обучающийся & \underline{\hspace{3cm}} & должность & ФИО \\\\
Руководитель & \underline{\hspace{3cm}}& должность & ФИО \\\\
\end{tabular}
}\\
\\
\\
\\
\\
\\
\\
\\
\\
\\
\\
\\
\\
\\
\\
\\
\\
\\
\\
\\
\\
\\
\\
\\
\\
\begin{center} Казань 2024 \end{center}
\thispagestyle{empty}
 

\newpage
\begin{center}
\renewcommand{\contentsname}{Содержание}
\tableofcontents
\newpage
\end{center}
 
\newpage
\section{ВВЕДЕНИЕ}

\begin{flushleft}
    Текст выравненный по левому краю с помощью команды flushleft\\
    Текст выравненный по левому краю с помощью команды flushleft\\
    Текст выравненный по левому краю с помощью команды flushleft\\
    Текст выравненный по левому краю с помощью команды flushleft\\
    Можно посмотреть в \cite{Reference1}\\
\end{flushleft}

\begin{flushright}
    Текст выравненный по правому краю с помощью команды flushright\\
    Текст выравненный по правому краю с помощью команды flushright\\
    Текст выравненный по правому краю с помощью команды flushright\\
    Текст выравненный по правому краю с помощью команды flushright\\
    Можно посмотреть в \cite{Reference2}\\
\end{flushright}

\begin{center}
    Текст выравненный по центру с помощью команды center\\
    Текст выравненный по центру с помощью команды center\\
    Текст выравненный по центру с помощью команды center\\
    Текст выравненный по центру с помощью команды center\\
    Можно посмотреть в \cite{Reference3}\\
\end{center}
\newpage
\section{Теоритическое исследование}
\subsection{Цель исследования и физическая постановка задачи}
\subsection{Обоснование направления исследования и его актуальность}
\subsection{Анализ и обобщение существующих исследований и опубликованных результатов}
\subsection{Методы исследования и применяемый математический аппарат}
\subsection{Содержание самостоятельно выполненных исследований и расчетови}
\subsection{Описание математической модели исследуемого процесса (явления)}
\newpage
\section{Разработка прикладного программного обеспечения}
\subsection{Обоснование и разработка алгоритмов, оценка их сложности}
\subsection{Обоснование выбранных средств программной реализации}
\subsection{Структура программного приложения (комплекса)}
\newpage
\section{Анализ и интерпритация результатов}
\subsection{Анализ и интерпретация результатов компьютерного моделирования вычислительного эксперимента}
\subsection{Оценка полноты решения поставленной задачи и возможные перспективы дальнейшей работы по теме}
\newpage
\section{СПИСОК ЛИТЕРАТУРЫ}
\renewcommand\refname{}

\begin{thebibliography}{5}
    \bibitem{Reference1}
    Автор1, Имя1.
    \emph{Статья или книга 1}.
    Год.
    
    \bibitem{Reference2}
    Автор2, Имя2.
    \emph{Статья или книга 2}.
    Год.

    \bibitem{Reference3}
    Автор3, Имя3.
    \emph{Статья или книга 3}.
    Год.

    \bibitem{Reference4}
    Автор4, Имя4.
    \emph{Статья или книга 4}.
    Год.

    \bibitem{Reference5}
    Автор5, Имя5.
    \emph{Статья или книга 5}.
    Год.
    \end{thebibliography}
\end{document}