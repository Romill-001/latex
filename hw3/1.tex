\documentclass[14pt]{article}
\usepackage[utf8x]{inputenc}
\usepackage[T2A]{fontenc}
\usepackage[russian,english]{babel}
\usepackage{amsmath}
\usepackage{cmap}
\usepackage{booktabs}
\usepackage{caption}
\usepackage{enumitem}
\usepackage{listings}
\usepackage{xcolor}
\usepackage{multirow}
\usepackage[a4paper, left=2cm, right=2cm, top=2cm, bottom=2cm]{geometry}
\renewcommand{\labelenumii}{\arabic{enumi}.\arabic{enumii}.}

\begin{document}
\textbf{Нумерованный список:}
\begin{enumerate}
    \item Первый элемент
    \item Второй элемент
    \item Третий элемент
\end{enumerate}

\textbf{Маркированный список:}
\begin{itemize}
    \item Первый элемент
    \item Второй элемент
    \item Третий элемент
\end{itemize}

\textbf{Словарь:}
\begin{description}
    \item[один] первый элемент
    \item[два] второй элемент
    \item[три] третий элемент   
\end{description}

\textbf{Кастомный список:}
\begin{enumerate}
    \item Первый элемент
    \begin{itemize}
        \item Вложенный маркированный список
    \end{itemize}
    \item Второй элемент
    \begin{enumerate}
        \item Вложенный Нумерованный список
    \end{enumerate}
    \item Третий элемент
    \begin{itemize}[label=$\bigodot$]
        \item Вложенный список с другим маркером
        \item 2
        \item 3
        \item 4
        \item 5
    \end{itemize}
    \item Четвёртый элемент
    \begin{itemize}[topsep=10pt, partopsep=10pt,itemsep = 10pt, parsep=10pt, left = 50pt]
        \item Вложенный список ...
        \item 1
        \item 2
        \item 3
        \item 4
        \item 5
    \end{itemize}
\end{enumerate}
\newpage
\begin{table}[h]
    \centering
    \caption*{Таблица 1 - подпись сверху
    }
    \begin{tabular}{|c|c|c|c|}
        \hline
        \multicolumn{2}{|c|}{Заголовок 1} & \multicolumn{2}{c|}{Заголовок 2} \\
        \hline
        Ячейка 1 & Ячейка 2 & Ячейка 3 & Ячейка 4 \\
        \hline
        \multirow{2}{*}{Объединенная строка} & \multicolumn{2}{|c|}{Объединенные ячейки} & Ячейка 5 \\
        \cline{2-4}
        & Ячейка 6 & Ячейка 7 & Ячейка 8 \\
        \hline
        Ячейка 9 & Ячейка 10 & Ячейка 11 & Ячейка 12 \\
        \hline
        \multirow{2}{*}{Объединенная строка} & Ячейка 13 & \multicolumn{2}{|c|}{Объединенные ячейки} \\
        \cline{2-4}
        & Ячейка 14 & Ячейка 15 & Ячейка 16 \\
        \hline
        Ячейка 17 & Ячейка 18 & Ячейка 19 & Ячейка 20 \\
        \hline
        
        \end{tabular}
    \caption*{Таблица 1 - подпись снизу}
\end{table}

\begin{tabbing}
    1 \= 2 \= 3 \= 4 \= 5 \\
    \kill
    \hspace{2cm} \= \hspace{2cm} \= \hspace{2cm} \= \hspace{2cm} \= \kill
    1 \> 2 \> 3 \> 4 \> 5 \\
    1 \> 2 \> 3 \> 4 \> 5 \\
    1 \> 2 \> 3 \> 4 \> 5 \\
\end{tabbing}
    
\end{document}