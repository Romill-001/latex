\documentclass[a4paper,12pt]{article}
\usepackage[utf8x]{inputenc}
\usepackage[T2A]{fontenc}
\usepackage[russian,english]{babel}
\usepackage{amsmath}
\usepackage{cmap}
\usepackage{booktabs}
\usepackage{caption}
\usepackage{enumitem}
\usepackage{xcolor}
\usepackage{setspace}
\usepackage[left=3cm, right=1.5cm, top=2cm, bottom=2cm]{geometry}
\renewcommand{\labelenumii}{\arabic{enumi}.\arabic{enumii}.}

\begin{document}

\begin{center}
\hfill \break
\textbf{\large{Министерство науки и высшего образования Российской Федерации\\
Федеральное государственное автономное образовательное\\
учреждение высшего образования}}
\\
\large{\textbf{«КАЗАНСКИЙ (ПРИВОЛЖСКИЙ) ФЕДЕРАЛЬНЫЙ УНИВЕРСИТЕТ»}}\\
\hfill \break
\large{ИНСТИТУТ ВЫЧИСЛИТЕЛЬНОЙ МАТЕМАТИКИ\\ И ИНФОРМАЦИОННЫХ ТЕХНОЛОГИЙ}\\
\hfill \break
\large{Кафедра прикладной математики и искусственого интеллекта}\\
\hfill\break
\hfill \break
\large{Направление подготовки: 01.03.04 – Прикладная математика}\\
\hfill \break
\hfill \break
\textbf{\large{ОТЧЁТ}}\\
\large{По дисциплине <<Современные издательские системы>>}\\
\large{на тему:}\\
\large{<<Набор и нумерация выключенных формул в \LaTeX>>}\\
\hfill \break
\hfill \break
\end{center}

\hfill \break
\begin{flushright}
			
    \large{Выполнил:}
    
    \large{студент группы 09-222}
    
    \large{Романов И. И.}
    
    \large{Проверил:}
    
    \large{Стехина К.Н.}
    
\end{flushright}
\vfill
\begin{center} \large{Казань, 2024 год} \end{center}
\thispagestyle{empty}

\newpage
\begin{center}
\renewcommand{\contentsname}{Содержание}
\fontsize{14}{1.15}\selectfont
\mdseries\selectfont{\tableofcontents}
\newpage
\end{center}
\setlength{\parindent}{1.25cm}
\newpage
\selectfont\onehalfspacing{
\section{Постановка задачи}
\hspace{1.25cm}Необходимо изучить и освоить на практике набор выклбюченных формул в системе LaTeX. Особое внимание при поиске информации необходимо уделить следующим пунктам:
\begin{itemize}
    \item Нумерованные и ненумерованные выключенные формулы.
    \item Организация ссылок и меток на формулы.
    \item Дроби, изменение шрифта для дробей.
    \item Интегралы, суммы, произведения и пределы для них.
    \item Скобки и знаки-ограничители. Изменение их размеров.
    \item Пробелы в формулах.
    \item Текст в формулах.
\end{itemize}

\section{Выключенные формулы}
\hspace{1.25cm}Для корректного набора формул, в частности выключенных, необходимо подклю\-чить пакеты \textsf{amsmath} и \textsf{mathtools}.

Выключенными формулами в LaTeX назваются формулы, которые находятся на отдельной строке в тексте. 
Обычно такой формат формул используется для подведения итогов вычислений или для выделения записей, на которые нужно обратить внимание.

Выключенные формулы бывают нумерованными и ненумерованными. Нумеро\-ван\-ные формулы имеют в правой части страницы порядковый номер.
Номера формул в документе начинаются с 1 и идут по порядку. Ненумерованные формулы же номера в правой части страницы не имеют.
Чтобы в документе добавить нумерованную формулу необходимо использовать окружение \textsf{equation}.
Для ненумерованных формул же необхо\-димо использовать окружение \textsf{equation*} либо \textsf{displaymath}

Пример выключенной нумерованной формулы:
\begin{equation}
    a^2 + b^2 = c^2
\end{equation}
\section{Ссылки и метки на формулы}
\hspace{1.25cm}Для ссылки на выкюченную формулу в LaTeX необходимо использовать команду \textsf{\textbackslash label\{eq:имя\_метки\}} внутри окружения после написанной формулы.
Чтобы в тексте документа сделать ссылку на формулу нужно использовать команду \textsf{\textbackslash eqref\{eq:имя\_метки\}}

\section{Дроби}
\hspace{1.25cm}В оформлении формул в LaTeX есть два основных стиля \textsf{\textbackslash displaystyle} (по умолча\-нию используется в выключенных формулах) и \textsf{\textbackslash textstyle}.
Их использование можно увидеть на примере дробей в LaTeX.

Для создания дроби в выклбюченных формулах в LaTeX необходимо использовать команду \textsf{\textbackslash frac\{...\}\{...\}}.
Для задания дроби в текстовом стиле (маленькая дробь) необхо\-димо использовать команду \textsf{\textbackslash tfrac\{...\}\{...\}}.
Для задания дроби в дисплейном стиле (большая дробь) необходимо использовать команду \textsf{\textbackslash dfrac\{...\}\{...\}}.

\section{Математические операции}
\hspace{1.25cm}Для набора математических операций, таких как сумма, интеграл, произведение,
нужно использовать соответственно команды \textsf{\textbackslash sum}, \textsf{\textbackslash int}, \textsf{\textbackslash prod}.

Для задания пределов в этих операциях можно использовать две конструкции: \textsf{\_\{min\} $\widehat{}$ \{max\}} и \textsf{\textbackslash limits\_\{min\} $\widehat{}$ \{max\}}.

При использованит первой конструкции для математической операции пределы будут помещены справа от знака операции,
при использовании же второй команды пределы будут помещены сверху и снизу от знака операции.

\section{Знаки-ограничители}
\hspace{1.25cm}В формулах LaTeX можно использовать различные скобки и знаки-ограничители.
Помимо стандартных круглых скобок, знаки-ограничители можно задать при помощи экранирования символов и специальных команд, например:\\
\textsf{( a ), [ b ], \textbackslash\{ c \textbackslash\}, | d |, \textbackslash| e \textbackslash|,
\textbackslash langle f \textbackslash rangle, \textbackslash lfloor g \textbackslash rfloor,
\textbackslash lceil h \textbackslash rceil,\\
/ j \textbackslash backslash, \textbackslash lbrack k \textbackslash rbrack}

\begin{equation*}
( a ), [ b ], \{ c \}, | d |, \| e \|,
\langle f \rangle, \lfloor g \rfloor,
\lceil h \rceil,
/ j \backslash, \lbrack k \rbrack
\end{equation*}

Изанчально все скобки и знаки-ограничители одного размера.
Это может пов\-ли\-ять на восприятие формулы, поэтому рекомендуется использовать парную команду \textsf{\textbackslash left ... \textbackslash right}.
Эта команда автоматически определяет размеры скобок в формулах. Задаются левая и правая скобка. Также можно задать скобку только с одной стороны.
Для этого вместо одной и скобок нужно использовать точку.

В некоторых случаях автоматически-определенные скобки могут быть избыточ\-ными, поэтому можно использовать команды для больших знаков-ограничителей:\\
\textsf{\textbackslash big( \textbackslash bigg( \textbackslash Big( \textbackslash Bigg(}

$$\big( \bigg( \Big( \Bigg($$

\section{Пробелы в формулах}
\hspace{1.25cm}Поскольку внутри формул в LaTeX пробелы не учитываются, можно использовать специальные команды, которые создают пробелы:
\begin{itemize}
    \item \textsf{\textbackslash quad} - выдаёт пробел размером равным размеру используемого шрифта, т.е., если шрифт размером 11п, то команда выдаст пробел размером в 11п.
    \item \textsf{\textbackslash qquad} - удвоенный quad
\end{itemize}
Однако эти команды могут быть избыточны, поэтому существуют альтернативы:
\begin{center}
    \begin{tabular}[h]{|c|c|c|}
    \hline
    Команда & Описание & Размер \\ \hline
    \textsf{\textbackslash ,} & Маленький пробел & 3/18 quad'а\\ \hline
    \textsf{\textbackslash :} & Средний пробел & 4/18 quad'а\\ \hline
    \textsf{\textbackslash ;} & Большой пробел & 5/18 quad'а\\ \hline
    \textsf{\textbackslash !} & Негативный пробел & -3/18 quad'а\\ \hline
    \end{tabular}
\end{center}

\section{Текст в формулах}
\hspace{1.25cm}Для добавления в формулу текста используются следующие команды:
\begin{itemize}
    \item \textsf{\textbackslash text \{...\}} - стандартная команда для добавления текста
    \item \textsf{\textbackslash mbox \{...\}} - альтернатива первой команде
    \item \textsf{\textbackslash textrm \{...\}} - использование стандартного шрифта
    \item \textsf{\textbackslash textbf \{...\}} - жирный текст
    \item \textsf{\textbackslash textit \{...\}} - курсивный текст
    \item \textsf{\textbackslash mathrm \{...\}} - прямой текст в формуле
\end{itemize}
\clearpage

\section{Выводы}
\hspace{1.25cm}В ходе выполнения работы мы научились набирать выключенные формулы в LaTeX.
Были изучены все наиболее важные аспекты работы с выключенными формулами, набор математических опреторов, скобок, пробелов, набор текста в формулах и другое.
\clearpage

\section{Список литературы}
\renewcommand\refname{}
\begin{thebibliography}{2}
    \bibitem{} 
    Р.В. Загретдинов, Ф.М. Албаев, Т.М.Гаврилова, С.Н. Перфилов. 
    {\bf Издательская система LaTeX. Краткое руководство} -- 
    Казань, 
    1994, 
    96 стр.
    \bibitem{} 
    {\bf Математические формулы в LaTeX} -- 
    Викиучебник. 
\end{thebibliography}
}
\end{document}