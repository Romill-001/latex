\documentclass[aspectratio=169]{beamer}
\usepackage[utf8]{inputenc}
\usepackage[T2A]{fontenc}
\usepackage[russian,english]{babel}
\usepackage{amsmath}
\usepackage{cmap}
\usepackage{caption}
\usepackage{xcolor}
\usepackage{listings}

\usetheme{Singapore} % Выбор темы презентации

\title{Набор и нумерация выключенных формул в \LaTeX} 
\author{Романов Илья 09-222}
\date{\today}
\logo{\large \LaTeX{}}


\begin{document}

\begin{frame}
    \titlepage 
\end{frame}

\logo{}


\begin{frame}{Содержание}
    \tiny
    \tableofcontents
\end{frame}

\section{Необходимые пакеты}
\begin{frame}{Необходимые пакеты}
    \begin{itemize}
        \item \bf{\textbackslash usepackage\{amsmath\}}
        \item \bf{\textbackslash usepackage\{mathtools\}}
    \end{itemize}
\end{frame}

\section{Необходимое окружение}
\begin{frame}{Необходимое окружение}
    \begin{minipage}[t]{0.45\textwidth}
        \begin{block}{Нумерованные формулы}
            \begin{itemize}
                \item \bf{equation}
            \end{itemize}
        \end{block}
    \end{minipage}

    \begin{minipage}[t]{0.45\textwidth}
        \begin{block}{Ненумерованные формулы}
            \begin{itemize}
                \item \bf{equation*} (требует подключения пакета amsmath)
                \item \bf{displaymath}
            \end{itemize}
        \end{block}
    \end{minipage}
\end{frame}

\section{Дроби}
\begin{frame}{Дроби}
    \begin{minipage}{0.3\textwidth}
        \bf{\textbackslash frac\{...\}\{...\}}
        
        \begin{equation*}
            text \quad \frac{1}{2} \quad text
        \end{equation*}   
    \end{minipage}
    \begin{minipage}{0.3\textwidth}
        \bf{\textbackslash tfrac\{...\}\{...\}}

        \begin{equation*}
            text \quad \tfrac{1}{2} \quad text
        \end{equation*}  
    \end{minipage}
    \begin{minipage}{0.3\textwidth}
        \bf{\textbackslash dfrac\{...\}\{...\}}

        $text \quad \dfrac{1}{2} \quad text$
    \end{minipage}
\end{frame}

\section{Математические операции}
\begin{frame}{Математические операции}
    \begin{itemize}
        \item Сумма - \textbackslash sum $\displaystyle\sum$
        \item Интеграл - \textbackslash int $\displaystyle\int$ 
        \item Произведение - \textbackslash prod $\displaystyle\prod$
    \end{itemize}
\end{frame}

\section{Пределы математических операций}
\begin{frame}{Пределы математических операций}
    \textbf{Способы объявления пределов:}
    \begin{itemize}
        \item Использование операции без конструкций – выводит на экран символ операции.
        \item Использование конструкции \_\{min\} $\wedge$ \{max\}.
        \item Использование команды \textbackslash limits\_\{min\} $\wedge$ \{max\}.
    \end{itemize}
\end{frame}

\section{Разница между \_\{min\} $\wedge$ \{max\} и \textbackslash limits\_\{min\} $\wedge$ \{max\}.}
\begin{frame}{Разница между \_\{min\} $\wedge$ \{max\} и \textbackslash limits\_\{min\} $\wedge$ \{max\}.}
    \bf
    \centering
    Использование \textbackslash limits\_\{min\} $\wedge$ \{max\}:
    \begin{equation*}
            \textstyle\sum\limits_{k=0}^{n}
    \end{equation*}

    Использование \textbackslash \_\{min\} $\wedge$ \{max\}:
    \begin{equation*}
            \textstyle\sum_{k=0}^{n}
    \end{equation*}    
\end{frame}

\section{Скобки и ограничители}
\begin{frame}{Скобки и ограничители}
    \textbf{Скобки в \LaTeX можно задавать как при помощи экранирования символов, так и при помощи различных команд:}

    \small
    ( a ), [ b ], \textbackslash\{ c \textbackslash\}, | d |, \textbackslash| e \textbackslash|,
    \textbackslash langle f \textbackslash rangle, \textbackslash lfloor g \textbackslash rfloor,\\
    \textbackslash lceil h \textbackslash rceil,\\
    / j \textbackslash backslash, \textbackslash lbrack k \textbackslash rbrack

    \centering
    $( a ), [ b ], \{ c \}, | d |, \| e \|,
    \langle f \rangle, \lfloor g \rfloor,
    \lceil h \rceil,
    / j \backslash, \lbrack k \rbrack
    $
\end{frame}

\section{Конструкция \textbackslash left ... \textbackslash right}
\begin{frame}{Конструкция \textbackslash left ... \textbackslash right}
    \textbf{Данная конструкция предназначена для автоматического определения размеров символов-ограничителей.}

    \small
    \textit{Примеры использования:}

    \textbackslash left(\textbackslash frac\{a + b\}\{d + c\}\textbackslash right) $\left(\dfrac{a + b}{d + c}\right)$

    \textbackslash left\{\textbackslash frac\{a + b\}\{d + c\}\textbackslash right\} $\left\{\dfrac{a + b}{d + c}\right\}$

    \textbackslash left[\textbackslash frac\{a + b\}\{d + c\}\textbackslash right] $\left[\dfrac{a + b}{d + c}\right]$

    \textbackslash left\{\textbackslash frac\{a + b\}\{d + c\}\textbackslash right. $\left\{\dfrac{a + b}{d + c}\right.$
\end{frame}

\section{Ручное определение размеров}
\begin{frame}{Ручное определение размеров}
    \textbf{Автоматическое изменение размера разделителей не всегда может приводить к желаемому результату, поэтому размеры разделителей можно определить самому с помощью команд-модификаторов:}
    \begin{minipage}{0.45\textwidth}
        \begin{itemize}
            \item \textbackslash big
            \item \textbackslash bigg
            \item \textbackslash Big
            \item \textbackslash Bigg
        \end{itemize}
    \end{minipage}
    \begin{minipage}{0.45\textwidth}
        \centering
        \textbackslash big( \textbackslash bigg( \textbackslash Big( \textbackslash Bigg(

        $\big( \bigg( \Big( \Bigg($
    \end{minipage}
\end{frame}

\section{Пробелы в формулах}
\begin{frame}{Пробелы в формулах}
    \textit{Всего есть 2 основные команды для набора пробелов:}
    \begin{itemize}
        \item \textbackslash quad - выдаёт пробел размером равным размеру используемого шрифта, т.е., если шрифт размером 11п, то команда выдаст пробел размером в 11п.
        \item \textbackslash qquad - удвоенный quad
    \end{itemize}
    \textit{Однако эти команды могут быть избыточны, поэтому существуют альтернативы:}
    \begin{center}
        \begin{tabular}[h]{|c|c|c|}
        \hline
        Команда & Описание & Размер \\ \hline
        \textbackslash , & Маленький пробел & 3/18 quad'а\\ \hline
        \textbackslash : & Средний пробел & 4/18 quad'а\\ \hline
        \textbackslash ; & Большой пробел & 5/18 quad'а\\ \hline
        \textbackslash ! & Негативный пробел & -3/18 quad'а\\ \hline
        \end{tabular}
    \end{center}
\end{frame}

\section{Текст в формулах}
\begin{frame}{Текст в формулах}
    \textbf{Для добавления в формулу текста используются следующие команды:}
    \begin{itemize}
        \item \textbackslash text \{...\} - стандартная команда для добавления текста
        \item \textbackslash mbox \{...\} - альтернатива первой команде
        \item \textbackslash textrm \{...\} - использование стандартного шрифта
        \item \textbackslash textbf \{...\} - жирный текст
        \item \textbackslash textit \{...\} - курсивный текст
    \end{itemize}
\end{frame}

\section{Ссылки и метки на формулы}
\begin{frame}{Ссылки и метки на формулы}
    Для создания метки на формулу используется команда \textbf{\textbackslash label\{eq:метка\}}.

    Далее в тексте можно использовать команду \textbf{\textbackslash eqref\{eq:метка\}}, которая выведет номер формулы в скобочка.

    Можно использовать и команду \textbf{\textbackslash ref\{eq:метка\}}, но в данном случае номер формулы будет без скобочек.
\end{frame}
\end{document}